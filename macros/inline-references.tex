%%%%%%%%%%%%%%%%%%%%%%%%%%%%%%%%%%%%%%%%%%%%%%%%%
% Bibliography Style Macros: Article Document
%%%%%%%%%%%%%%%%%%%%%%%%%%%%%%%%%%%%%%%%%%%%%%%%%

\newcommand{\jcpARef}[6]{
        \noindent #1. \textit{#2} \textbf{#3}, #4 (#5) \href{https://doi.org/#6}{(doi: #6)}}
        % 1: Authors, First initial(s), Last Name, comma separated
        % 2: Journal abbreviation
        % 3: volume
        % 4: page number
        % 5: year
        % 6: doi

\newcommand{\prlARef}[5]{
        \noindent #1. \textit{#2} \textbf{#3}, #4 (#5).}
        % 1: Authors, First initial(s), Last Name, comma separated
        % 2: Journal abbreviation
        % 3: volume
        % 4: page number
        % 5: year

\newcommand{\prlAHRef}[6]{
        \noindent #1. \href{<#6>}{<\textit{#2} \textbf{#3}, #4 (#5).>}}
        % 1: Authors, First initial(s), Last Name, comma separated
        % 2: Journal abbreviation
        % 3: volume
        % 4: page number
        % 5: year
        % 6: URL to paper

\newcommand{\ACSbookARef}[5]{
        \noindent #1. \textit{#2}. #3: #4, #5.}
        % 1: Authors, ACS format
        % 2: Book title
        % 3: Publisher
        % 4: Place of Publication
        % 5: Year of publication

\newcommand{\ACSbookchapterARef}[8]{
        \noindent #1. ``#2,'' in \textit{#3}, Vol. #4, #5 (#6). \href{#8}{(doi: #7)}}
        % 1: Authors, ACS format
        % 2: Chapter Title
        % 3: Book title
        % 4: Volume
        % 5: Pages
        % 6: Year
        % 7: DOI
        % 8: DOI link

\newcommand{\APSbookARef}[5]{
        \noindent #1, \textit{#2} (#3, #4, #5).}
        % 1: Authors, APS format
        % 2: Book title
        % 3: Publisher
        % 4: Place of Publication
        % 5: Year of Publication
  
\newcommand{\compendARef}[6]{
        \noindent\textit{#1}; #2, Eds.; #3; #4: #5, #6.}
        % 1: Title of book
        % 2: Editors, ACS author format
        % 3: Book series title & number
        % 4: Publisher
        % 5: Place of Publication
        % 6: year of publication

\newcommand{\coursematARef}[3]{
        \noindent{#1, ``#2.''  #3 course materials: available on T-Square.}}
    % 1: Authors, APS format
    % 2: Title of materials
    % 3: Course abbreviation & number, e.g., CHEM 6491

%%%%%%%%%%%%%%%%%%%%%%%%%%%%%%%%%%%%%%%%%%%%%%%%%
% Bibliography Style Macros: Beamer Presentation
%%%%%%%%%%%%%%%%%%%%%%%%%%%%%%%%%%%%%%%%%%%%%%%%%

\newcommand{\jcpBRef}[4]{
        \textit{#1} \textbf{#2}, #3 (#4)}
        % 2: Journal abbreviation
        % 3: volume
        % 4: page number
        % 5: year
        % 6: doi

\newcommand{\prlBRef}[4]{
        \textit{#1} \textbf{#2}, #3 (#4).}
        % 2: Journal abbreviation
        % 3: volume
        % 4: page number
        % 5: year

\newcommand{\prlBHRef}[5]{
        \href{<#5>}{<\textit{#1} \textbf{#2}, #3 (#4).>}}
        % 2: Journal abbreviation
        % 3: volume
        % 4: page number
        % 5: year
        % 6: URL to paper

\newcommand{\ACSbookBRef}[4]{
        \textit{#1}. #2: #3, #4.}
        % 2: Book title
        % 3: Publisher
        % 4: Place of Publication
        % 5: Year of publication

\newcommand{\APSbookBRef}[4]{
        \textit{#1} (#2, #3, #4).}
        % 2: Book title
        % 3: Publisher
        % 4: Place of Publication
        % 5: Year of Publication
  
\newcommand{\compendBRef}[6]{
        \noindent\textit{#1}; #2, Eds.; #3; #4: #5, #6.}
        % 1: Title of book
        % 2: Editors, ACS author format
        % 3: Book series title & number
        % 4: Publisher
        % 5: Place of Publication
        % 6: year of publication

\newcommand{\coursematBRef}[3]{
        \noindent{#1, ``#2.''  #3 course materials: available on T-Square.}}
    % 1: Authors, APS format
    % 2: Title of materials
    % 3: Course abbreviation & number, e.g., CHEM 6491

%%%%%%%%%%%%%%%%%%%%%%%%%%%%%%%%%%%%%%%%%%%%%%%%%
% Bibliography Style Macros: CV
%%%%%%%%%%%%%%%%%%%%%%%%%%%%%%%%%%%%%%%%%%%%%%%%%
\newcommand{\PaperEntry}[7]{
    \noindent ``#1,'' #2, \textit{#3} \textbf{#4}, #5 (#6) \href{https://dx.doi.org/#7}{(doi: #7)}}
    % 1: Paper Title
    % 2: Author List
    % 3: Journal Abbreviation
    % 4: Volume number
    % 5: Page number
    % 6: Year
    % 7: doi

\newcommand{\SubmittedPaperEntry}[3]{
    \noindent ``#1,'' #2, \textit{#3} (Submitted)}
    % 1: Paper Title
    % 2: Author List
    % 3: Journal Abbreviation

\newcommand{\SubmittedPaperwRXIV}[5]{
    \noindent ``#1,'' #2, \textit{#3} (Submitted) \href{#5}{ChemRxiv: #4}}
    % 1: Paper Title
    % 2: Author List
    % 3: Journal Abbreviation
    % 4: DOI
    % 5: DOI URL for linking

\newcommand{\InprepPaperEntry}[2]{
    \noindent ``#1,'' #2. (In preparation)}
    % 1: Paper Title
    % 2: Author list

\newcommand{\ArxivEntry}[3]{
    \noindent #1, ``\href{http://arxiv.org/abs/#3}{#2}", \textit{{cond-mat/}#3}.}

\newcommand{\ChemRxivEntry}[5]{
    \noindent ``{\bf #1},'' #2, (#3). \href{#5}{(doi: #4)}
    % 1: Preprint Title
    % 2: Author list, JCP formatting
    % 3: Year
    % 4: DOI
    % 5: DOI URL for linking
    }

\newcommand{\BookEntry}[4]{
    \noindent #1, ``\href{#3}{#4}", \textit{#3}.}

\newcommand{\FellowEntry}[3]{
        \noindent #1, #2. (#3)}
        % 1: Fellowship title
        % 2: Funding agency

\newcommand{\PresEntry}[6]{
    \noindent #1, #2. #3: ``#4" (#5) #6}

\newcommand{\TalkEntry}[7]{
    \noindent ``\textbf{#1},'' Contributed talk at the #2. #3, \##4. (#5)
    \begin{description}
    \item[#6] #7
    \end{description}}
    % 1: Title
    % 2: Meeting
    % 3: Symposium Title
    % 4: Talk #
    % 5: Date, mm/yyyy
    % 6-7: Award?
        % 6: Place:
        % 7: Award Title

%\newcommand{\AIPConfPresWithEditor}[10]{
%    \noindent #1, #2. In: {\em #3: Proceedings of the #4, #5}, Ed. by #6. (#7, #8. #9), #10.
%    }
    % 1: Presentation authors, presenting author underlined
    % 2: Presentation title
    % 3: Title of collected work
        % Example (Presentation at ACS): #3 = {Book of Abstracts}
    % 4: Name of Meeting
        % Example (Presentation at ACS): #4 = {Spring 2024 National Meeting of the American Chemical Society}
        % Example (MERCURY): #4 = {22$^{\rm nd}$ MERCURY Conference on Computational Chemistry}
    % 5: Location and Date of Conference
        % Example (ACS Spring 2024): #5 = {New Orleans, LA, USA, March 17-21 2024}
    % 6: Conference proceedings editor (same name format as authors)
    % 7: Publisher
        % Example (ACS): #7 = {ACS Publishing}
    % 8: Place of Publication
        % Example (ACS): #8 = {Washington, DC}
    % 9: Year of Publication (same as year of presentation)
    % 10: Presentation number/Paper ID
        % Example (ACS): #10 = {\#3979135}

\newcommand{\AIPConfPres}[9]{
    \noindent #1, #2. In: {\em #3: Proceedings of the #4, #5}. (#6, #7. #8), #9.
    }
    % 1: Presentation authors, presenting author underlined
    % 2: Presentation title
    % 3: Title of collected work
        % Example (Presentation at ACS): #3 = {Book of Abstracts}
    % 4: Name of Meeting
        % Example (Presentation at ACS): #4 = {Spring 2024 National Meeting of the American Chemical Society}
        % Example (MERCURY): #4 = {22$^{\rm nd}$ MERCURY Conference on Computational Chemistry}
    % 5: Location, Month, Dates, Year of Conference
        % Example (ACS Spring 2024): #5 = {New Orleans, LA, USA, March 17-21 2024}
    % 6: Publisher
        % Example (ACS): #7 = {ACS Publishing}
    % 7: Place of Publication
        % Example (ACS): #8 = {Washington, DC}
    % 8: Presentation number/Paper ID
        % Example (ACS): #10 = {\#3979135}

\newcommand{\PosterEntry}[7]{
    \noindent ``\textbf{#1},'' Contributed poster at the #2. #3, \##4. (#5)
    \begin{description}
    \item[#6] #7
    \end{description}}
    % 1: Title
    % 2: Meeting
    % 3: Symposium Title
    % 4: Talk #
    % 5: Date, mm/yyyy
    % 6-7: Award?
        % 6: Place:
        % 7: Award Title

\newcommand{\SeminarEntry}[5]{
    \noindent ``\textbf{#1},'' #2, #3: #4. (#5)}
    % 1: Title
    % 2: Institution
    % 3: Department
    % 4: Colloquium Name
    % 5: Date, mm/yyyy

\newcommand{\VolunteerEntry}[4]{
    \noindent #1, #2. (#3)\\
        \noindent Description: #4.}

\newcommand{\ThesisEntry}[5]{
    \noindent #1 -- #2 #3 ``#4" \textit{#5}}

\newcommand{\CourseEntry}[3]{
    \noindent \item{#1: \textbf{#2} \\ #3}}
