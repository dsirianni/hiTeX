%%
%%
%% NSF requires minimum of 10pt font with margins on all sides of at least
%% 2.5cm.  No more than 15 characters (including spaces) per 2.5cm (avg).
%% No more than 6 lines of type w/in a vertical space of 2.5cm. 
%%
%%

\documentclass[11pt]{article}
\usepackage{times}
%\usepackage{html}
\usepackage{overcite}
\usepackage{graphicx}

\setlength{\topmargin}{0cm}
\setlength{\evensidemargin}{0cm}
\setlength{\oddsidemargin}{0cm}
\setlength{\headsep}{0cm}
\setlength{\headheight}{0cm}
\setlength{\marginparwidth}{0cm}
\setlength{\marginparpush}{0pt}
\setlength{\textwidth}{6.4in}
\setlength{\textheight}{8.7in}
\setlength{\footskip}{20pt}
\parskip 3pt


%%%%%%% DEFINITIONS, ETC %%%%%%%%%%%
\def\ket#1{| #1 \rangle}
\def\bra#1{\langle #1 |}
\def\spinor#1{\chi_{#1}}
\def\creat#1{a_{#1}^{\dagger}}
\def\annih#1{a_{#1}}

\begin{document}

\thispagestyle{empty}
\setcounter{secnumdepth}{4}
\renewcommand{\thesubsection}{\Alph{subsection}}
\renewcommand{\thesubsubsection}{\arabic{subsubsection}}
                                                                                
\begin{center}
\begin{large}
{\bf PROJECT SUMMARY} \\
\end{large}
\end{center}
\hrule

\vspace*{0.1in}

\hspace*{-0.28in} {\bf Overview:} \\ 

% We propose to develop much faster
% ways to accurately evaluate the strength of non-bonded interactions,
% which govern biomolecular structure, the structure of
% organic crystals, and the binding of ligands by proteins.
% In previous work, we have produced methods and software
% to compute intermolecular interactions with
% of symmetry-adapted perturbation theory (SAPT) based on
% many-body perturbation theory or coupled-cluster theory.
% At the most accurate levels, SAPT is of coupled-cluster with perturbative
% triples [CCSD(T)] quality.  Thanks to favorable error cancellation,
% even the lowest-level treatment, SAPT0, typically provides reliable
% interaction energies.  As a bonus, SAPT provides each energy component
% separately: electrostatics, induction/polarization, London disperson
% interactions, and exchange-repulsion.
% In large, complex systems, even the less expensive SAPT methods are
% too computationally expensive to be practical.  (1) We will
% speed up SAPT0 dramatically by replacing the rate-determining
% dispersion step with very fast estimates:
% (a) semi-empirical damped pairwise-atomic terms, with
% damping parameters carefully fitted to large training sets (previous
% parameters from the literature do not appear to perform well), or (b)
% estimates determined directly from machine learning (ML).  (2) We will
% develop fast ML models of {\em all four} interaction
% energy components (electrostatics, dispersion, induction, exchange),
% based on large training data sets.  (3) We will
% develop advanced physics-based intermolecular potentials, whose
% many parameters are obtained through ML.  (4) We will
% increase the accuracy of low-order SAPT0, using delta-learning
% techniques to obtain a ML model of the difference between
% SAPT0 and ``gold standard'' CCSD(T).

\hspace*{-0.28in} {\bf Intellectual Merit:} \\ 

%Intermolecular
%interactions pose some particular challenges to machine learning: (1)
%selection of suitable training and validation data is less obvious, (2)
%standard descriptors appear non-ideal for intermolecular interactions,
%(3) we would like to retain the energy components as well as the
%total interaction energy, leading to a multi-task learning problem.
%Our proposal is aimed at overcoming these challenges.  At the same
%time, it is not clear whether ML models of intermolecular interaction
%energies can outcompete high-quality, next-generation force-field
%model potentials.  The biggest problem with the latter is that they
%can require too many parameters for their extension to general organic
%molecules to be practical;  however, parameterization problem might be
%neatly solved by using ML to obtain the requisite parameters.  We will
%pursue both pure-ML and physics+ML approaches in an internal competition,
%using common test sets, which will highlight the strength and weaknesses
%of both approaches.  Finally, we
%have a unique capability to partition SAPT energies into pairwise-atomic
%contributions, through our A-SAPT method.  Because Behler-Parrinello
%neural networks model properties on an atomic basis, A-SAPT data is {\em
%precisely} what we are attempting to learn.  This should make training
%much easier and require much less data than indirectly training to the
%{\em sums} of atomic contributions in van der Waals dimers, and could be
%a significant innovation in applying ML to intermolecular interactions.

\hspace*{-0.28in} {\bf Broader Impacts of the Proposed Work:} \\

%Proposed methods, if successful, will be 
%released as open-source software to aid other researchers.
%Having various levels of accuracy vs computational speed will enable a
%variety of applications.   The faster and more accurate SAPT0 methods
%will be helpful in understanding longer-range contacts between ligands
%and proteins.  The fastest methods will be suitable for rapid screening
%of ligand binding or candidate structures of organic crystals.
%Retaining the ability to compute energy components will provide an
%additional level of insight and interpretability, which has proven
%very helpful in previous applications of SAPT.  The PI will continue
%to organize summer bootcamps in data science, and will update
%his popular YouTube videos on quantum chemistry.  The project will
%contribute to the training of postdocs, graduate students,
%and undergraduates in quantum chemistry methods, data science, machine
%learning, and software development.

\end{document}

