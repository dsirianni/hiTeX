%%
%%
%% NSF requires minimum of 10pt font with margins on all sides of at least
%% 2.5cm.  No more than 15 characters (including spaces) per 2.5cm (avg).
%% No more than 6 lines of type w/in a vertical space of 2.5cm.
%%
%%

\documentclass[12pt]{article}
\usepackage{epsfig}
%\usepackage{html}
% citesort seems obsolete now
%\usepackage{citesort}
\usepackage{overcite}
\usepackage{amsmath}
\usepackage{amssymb}
\usepackage{graphicx}
\usepackage{times}
\usepackage{wrapfig}
\usepackage{threeparttable}
%\usepackage[normal,footnotesize]{caption}
\usepackage[normal,footnotesize]{}
\usepackage{amssymb}
\usepackage{xcolor}

\setlength{\topmargin}{0cm}
\setlength{\evensidemargin}{0cm}
\setlength{\oddsidemargin}{0cm}
\setlength{\headsep}{0cm}
\setlength{\headheight}{0cm}
\setlength{\marginparwidth}{0cm}
\setlength{\marginparpush}{0pt}
\setlength{\textwidth}{6.5in}
\setlength{\textheight}{8.7in}
\setlength{\footskip}{20pt}
\addtolength{\abovecaptionskip}{-3mm}
\parskip 3pt

\renewcommand{\thefootnote}{\fnsymbol{footnote}}

\begin{document}

\setcounter{secnumdepth}{4} \renewcommand{\thesubsection}{\Alph{subsection}}
\renewcommand{\thesubsubsection}{\arabic{subsubsection}}

\subsection*{Project Description}

\subsubsection{Results from Prior NSF Support}

{\bf Intellectual Merit:} \\

[[ Broad-strokes overview of research program, 1 paragraph ]]\\

\paragraph{Specific Aim 1}

[[ Several paragraphs on one specific aim of the research program ]]\\

\paragraph{Specific Aim 2}

[[ Several paragraphs on one specific aim of the research program ]]\\

\paragraph{Specific Aim 3}

[[ Several paragraphs on one specific aim of the research program ]]\\

\vspace*{-4mm} \paragraph{Broader Impacts.}

[[ 2-3 paragraph statement of broader impacts of research program ]]\\

%% Example: David's broader impacts from 2019 proposal
% Computational simulations have joined theory and experiment as one of the
% primary modes of scientific discovery.\cite{DOE:1999} With our collaborators,
% we have released {\sc Psi4}, a fully open-source quantum chemistry code with a
% wide variety of features.\cite{Turney:PSI4, Parrish:2017:3185}  The code is
% efficient for most tasks, works well on multi-core machines, and is built on a
% modular infrastructure designed to make it easy to incorporate new theoretical
% advances and algorithms.  Much of the code was re-written to take advantage of
% density fitting algorithms, resulting in very fast Hartree--Fock, DFT, MP2,
% SAPT, and CCSD(T) codes.  We have placed a major emphasis on interoperability,
% designing the code to be easy to interface with, and writing interfaces to
% numerous quantum chemistry libraries and modules developed by
% others.\cite{Parrish:2017:3185} We have also emphasized user-friendliness and
% automation of common tasks.  Indeed, a user can invoke a focal-point,
% counterpoise-corrected, or complete-basis-set extrapolated computations with a
% single line of input.  {\sc Psi4} has been downloaded more than 102,000 times
% as of the end of 2018 (more than half these downloads are from 2018).  The two
% {\sc Psi4} papers \cite{Turney:PSI4, Parrish:2017:3185} have more than 860
% citations between them.  In 2018, we released {\sc
% Psi4NumPy},\cite{Smith:2018:3504} an interface making it much easier to rapidly
% develop quantum chemistry code, using a programmer-friendly Python/NumPy
% environment undergirded by the {\sc Psi4} libraries.
% 
% Prior NSF funding of our research program has contributed to the training of
% numerous undergraduates, graduate students, postdocs, and visiting faculty.
% Graduate students from the group have won awards including NSF and DOE graduate
% fellowships, the Foresight Institute Nanotechnology Student of the Year Award,
% and the Iota Sigma Pi Anna Louise Hoffman Award.  Group alumni include Prof.
% Eugene DePrince (Florida State University), Prof. Ashley Ringer McDonald
% (CalPoly), Prof. Edward Valeev (Virginia Tech), Prof. Mutasem Sinnokrot
% (Khalifa University).
% 
% NSF funding has also continued to support the PI's educational efforts
% to make quantum chemistry more accessible to students.  His web
% notes and YouTube videos on quantum chemistry continue to be popular.
% The PI has created the {\sc Psi4}Education
% consortium,\cite{Fortenberry:2015:PSI4Education} consisting primarily
% of faculty at small colleges, to develop a series of computational
% chemistry laboratory modules that all use free software (primarily
% {\sc Psi4} and WebMO as its graphical front-end).

\subsubsection{Proposed Research}

\paragraph{Project Goals.}

[[ 1 paragraph statement of proposed research goals, broadly ]]\\

\paragraph{Project 1 Overview}

[[ Several paragraph description of preliminary results \& proposed work
towards completion of proposed Project 1 ]]\\

\paragraph{Project 2 Overview}

[[ Several paragraph description of preliminary results \& proposed work
towards completion of proposed Project 2 ]]\\

{\em Project 2 subheading} This is where a specific detail is expanded upon
for Project 2\\

{\em Challenge 1: Something something} This is where a particular challenge
that will be addressed by the proposal is expanded upon.
%% Figures should accompany these projects, and be included via wrapfigure
%\begin{wrapfigure}{r}{0.33\textwidth}
%\includegraphics[scale=0.65]{multitarget_plot.png}
%\caption{\small{Validation errors of BPNNs trained on SSI and
%configurations of 128 H-bonded dimers.
%}
%\label{fig:multitarget_plot}
%}

\paragraph{Project 3 Overview}

[[ Several paragraph description of preliminary results \& proposed work
towards completion of proposed Project 3 ]]\\

\vspace*{-4mm} \paragraph{Broader Impacts of the Proposed Work.}

%% David's proposed broader impacts from 2019 NSF prop
% The proposed methods development work will assist in the training of
% young researchers who will be capable of creating the next wave of
% computational tools.  Improved SAPT models will be folded into the
% {\sc Psi4} package, developed by us and our collaborators as free,
% open-source software.  {\sc Psi4} has been growing rapidly in popularity
% among both users and developers, with the SAPT capabilities developed at
% Georgia Tech being a key feature.  Pure-ML and force-field models will
% be released as stand-alone, open-source packages.  The proposed
% developments will provide the computational chemistry community a
% powerful new set of tools for computing intermolecular interactions
% reliably and with unprecedented speed.
% 
% Based on positive feedback from students, the PI proposes to create new and
% updated YouTube videos on quantum chemistry, including an ``Introduction
% to Quantum Chemistry'' video,
% and an introduction to the basics of SAPT for students and computational
% chemists not familiar with the method or software.  The PI also proposes
% to continue to develop and organize a series of successful summer
% ``bootcamps'' in data science.  He has organized two 10-hour bootcamps
% for Georgia Tech graduate students, drawn from all disciplines, with the
% time split between lecture and hands-on activities.  During summer 2019,
% he organized a more intensive, 22-hour bootcamp for 80 students drawn
% from Georgia Tech, Morehouse, Spelman, Agnes Scott, and Kennesaw State,
% again organized as a mix of lecture and hands-on activities.  84\% of
% the respondents to the post-bootcamp survey said they planned to use the
% material they learned in their own education and research projects. The
% PI described some of his group's work in machine learning and chemistry
% during the final, ``domain day'' of the bootcamp, and several students
% indicated in the assessment surveys that this was their favorite topic.
% The current proposal would provide the PI much additional material from
% which he could illustrate domain applications in the bootcamp.
% 
% Because undergraduates rarely have significant exposure to theoretical
% chemistry, funds are requested for one undergraduate researcher to
% assist with the proposed work.  Undergraduate research experiences can
% be very helpful in attracting students to theoretical chemistry (this
% was the case for the PI), and undergraduate researchers from our group
% have gone on to join chemistry PhD programs at universities including
% UT Austin, Cornell, U.~Georgia, Berkeley,
% Emory, and Notre Dame.  Former undergraduates Ryan
% Steele and Ryan Lively are now faculty members in Chemistry at Utah,
% and in Chemical Engineering at Georgia Tech, respectively.
% Brazilian exchange student Leonardo dos Anjos Cunha worked with
% the Sherrill group from 2015-2016
% and is now a graduate student in the Head-Gordon
% group at Berkeley.  German exchange student Marvin Lechner worked
% with the group in 2016 and is now a graduate student in the group of
% Frank Neese at the University of Bonn.  Undergraduate Nick Petosa is
% now earning a master's degree in Computer Science at Georgia Tech.
% Undergraduate Mike Zott won a poster award at the 2017 meeting of the
% Southeastern Theoretical Chemistry Association (SETCA) in 2017, and
% is now a graduate student in Chemistry at CalTech.  We will continue
% providing such educational opportunities in the renewal period.


%%% Bibliography %%%
\newpage
\setcounter{page}{1}
\bibliographystyle{nsf}
%\bibliography{jrncodes,mainbib-ds}
\bibliography{refs}

\end{document}



