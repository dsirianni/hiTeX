%%
%%
%% NSF requires minimum of 10pt font with margins on all sides of at least
%% 2.5cm.  No more than 15 characters (including spaces) per 2.5cm (avg).
%% No more than 6 lines of type w/in a vertical space of 2.5cm. 
%%
%%

\documentclass[12pt]{article}
\usepackage{epsfig}
%\usepackage{html}
%\usepackage{citesort}
\usepackage{overcite}
\usepackage{graphicx}
\usepackage{amsmath}
\usepackage{amssymb}
\usepackage{verbatim}
\usepackage{times}
\usepackage{wrapfig}
%\usepackage[normal,footnotesize]{caption}
%\usepackage[normal,footnotesize]{}
\usepackage[small]{caption}

\setlength{\topmargin}{0cm}
%\setlength{\topmargin}{2.0cm}
\setlength{\evensidemargin}{0cm}
\setlength{\oddsidemargin}{0cm}
\setlength{\headsep}{0cm}
\setlength{\headheight}{0cm}
\setlength{\marginparwidth}{0cm}
\setlength{\marginparpush}{0pt}
\setlength{\textwidth}{6.5in}
\setlength{\textheight}{8.7in}
\setlength{\footskip}{20pt}
\addtolength{\abovecaptionskip}{-3mm}
%\parskip 3pt

\begin{document}

\begin{center}
{\large \bf Georgia Tech: Facilities, Equipment, and Other Resources}
\end{center}

\vspace*{0.5cm}

\noindent
{\bf Laboratory / Office Space:} \\
Research will be conducted in room 4202 of the Molecular Science \&
Technology (MoSE) building at Georgia Tech.  This room, shared by the
Sherrill and McDaniel groups, provides 1620 sq.~ft.~of space, with ample
electric power and cooling to support high-end computer workstations
at every desk.  Space is available for all of the personnel assigned
to this project.  Dr.~Sherrill's office is located two floors down from
the laboratory.
\\

\noindent
{\bf Sherrill Group Computational Resources:} \\
The Sherrill group has exclusive use of a 15-node cluster for medium-scale
applications and testing.  This cluster consists of 6-core Intel i7 processors,
each with 64GB of RAM and 9 TB of RAID0 scratch space.  
\\

\noindent
All desks in the MoSE 4201 laboratory are equipped with
Linux desktop workstations with Intel i7 6-core processors and 64 or 128 
GB of RAM.
\\

\noindent
{\bf Georgia Tech Chemistry Cluster:} \\
In 2017, a Chemistry cluster was established with 24 nodes, each with
256 GB RAM, 20 cores (Intel Xeon E5-2640 v4 @ 2.4 GHz), 5 TB of RAID0
scratch disk, and networked via quad-data-rate Infiniband.  Additional,
similarly-equipped nodes have been added over time for a current total
of 37 nodes.  These resources are shared by the Sherrill, Br{\'e}das,
McDaniel, and Kretchmer groups at Georgia Tech.
An additional 6 nodes
are available for exclusive use by the Sherrill group.  These nodes
are housed and maintained by the Partnership for an Advanced Computing
Environment at Georgia Tech.
\\

\noindent {\bf Georgia Tech Hive Cluster:} \\ Dr.~Sherrill is one of
5 co-PI's on an NSF MRI grant that was used to purchase Hive,
a 483-node cluster with 24 cores per node (Intel Xeon Gold 6226
processors) in August, 2019.  Nodes are networked with 10GbE EDR
(100 Gb/s) Infiniband.  16 of the nodes are equipped with 4 Nvidia
V100 GPU's each, which may be helpful in accelerating the training of
machine learning models.  For the generation of high-quality training
data that may require significant I/O [CCSD(T) or higher-order SAPT], 16
of the nodes contain local SAS disks (4 x 1.8 TB RAID0, 7.3 TB usable),
and 16 of the nodes contain NVMe-interfaced local disks (5 x 1.5 TB
RAID0, 7.3 TB usable).  Ample time will be
available for any data generation and machine learning training
applications that may go beyond what can be supported by the Sherrill
group or Chemistry clusters.  This cluster is housed and maintained by
the Partnership for an Advanced Computing Environment at Georgia Tech.
\\

\noindent
{\bf Georgia Tech FoRCE cluster:} \\
In addition to the above resources, Dr.~Sherrill also has access to the
Georgia Tech FoRCE cluster, a heterogeneous mixture of 356 AMD Opteron
and Intel Xeon compute nodes (total of 9820 cores) connected by DDR to
QDR Infiniband.  The Sherrill group is typically able to use several
dozen cores at a time.  These nodes are housed and maintained by the
Partnership for an Advanced Computing Environment at Georgia Tech.
\\

\noindent
{\bf Collaborations:} \\
We are fortunate to have a large community of researchers at Georgia
Tech interested in machine learning and its applications to science
and engineering.  Dr.~Le Song, Associate Professor in the School of
Computational Science and Engineering at Georgia Tech, and Associate
Director of our Center for Machine Learning, has agreed to consult
with us on the potential application of his previously developed
message-passing descriptors for atomic environments (see attached
letter of collaboration).
\\

\end{document}
 

