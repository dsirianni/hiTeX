%GT-CSIP Beamer Theme.

\documentclass[compress]{beamer}
\mode<presentation>

\usetheme{Luebeck}
\hypersetup{pdfpagemode=FullScreen} % makes your presentation go automatically to full screen

% define your own colors:
\definecolor{gold}{rgb}{260,195,0}

% could choose different themes for the "inside" and "outside"
% \usepackage{beamerinnertheme_______}
% inner themes include circles, default, inmargin, rectangles, rounded
% outer themes include default, infolines, miniframes, shadow, sidebar, smoothbars, smoothtree, split, tree
\useinnertheme{default}
\useoutertheme{default}

% To have the same footer on all slides
\setbeamertemplate{footline}{\tiny{} } % makes the footer empty
\setbeamertemplate{headline}[default]
\setbeamertemplate{navigation symbols}{}
\setbeamertemplate{blocks}[framed]

% Make GT-CSIP footer logo scheme
\setbeamercolor{structure}{fg = blue!40!gold,bg=blue!40!gold}
\setbeamercolor*{palette primary}{use=structure,fg=white,bg=blue!40!gold}
% Create footer with frame number and GTRI logo
\usefoottemplate{\vbox{
	\hspace{-24px}			
	\insertframenumber \, of \inserttotalframenumber \hfill
	\includegraphics[height=0.8cm] {gt}
	\hspace{8px}
	\includegraphics[height=0.8cm] {csip_new_logo}
	\hspace{-24px}
	\vspace{4px}
}}

% include packages
\usepackage{amsmath} %helpful package always
\usefonttheme{professionalfonts} %use regular CM fonts for math mode
\bibliographystyle{apalike} %
\graphicspath{{images/}}

%  Outline at each section
\AtBeginSection[] {
  \begin{frame}
   \frametitle{Outline}
    \tableofcontents[currentsection]
    \addtocounter{framenumber}{-1}
  \end{frame}
}

%  Could have outline at each subsection
%\AtBeginSubsection[] {
%  \begin{frame}
%   \frametitle{Outline}
%    \tableofcontents[currentsubsection]
%    \addtocounter{framenumber}{-1}
%  \end{frame}
%}

%  Set title information
\title{The Title}
\author{The Author}
\date{\today}

\begin{document}

%  Make Title Page
{\usebackgroundtemplate{\includegraphics[width=\paperwidth,height= \paperheight]{tower}}
\frame{
\setcounter{framenumber}{0}
\titlepage
}}

%  Make beginning outline
\frame{\frametitle{Outline}
\tableofcontents
}

%  Begin content
\section{The First Section}
\frame{\frametitle{The Frame Title}
Some Words
\begin{itemize}
\item The First Item
\begin{equation}
e^{j\varphi}=\cos{\varphi}+j\sin{\varphi}
\end{equation}
\item The Second Item
\item The Third Item
\end{itemize}
}

\section{The Second Section}
\frame{\frametitle{The Frame Title}
Some Words
\begin{itemize}
\item The First Item
\begin{equation}
e^{j\varphi}=\cos{\varphi}+j\sin{\varphi}
\end{equation}
\item The Second Item
\item The Third Item
\end{itemize}
}

\section{The Third Section}
\frame{\frametitle{The Frame Title}
Some Words
\begin{itemize}
\item The First Item
\begin{equation}
e^{j\varphi}=\cos{\varphi}+j\sin{\varphi}
\end{equation}
\item The Second Item
\item The Third Item
\end{itemize}
}

\end{document}
