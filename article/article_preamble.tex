%%%%%%%%%%%%%%%%%%%%%%%%%%%%%%%%%%%%%%%
% Page Formatting
%%%%%%%%%%%%%%%%%%%%%%%%%%%%%%%%%%%%%%%

\documentclass[fleqn,oneside,12pt]{article}

% Standard Page Dimensions
\usepackage[top=1in, bottom=1in, left=1in, right=1in]{geometry}

% Custom Page Dimensions
%\usepackage{blindtext}
%\usepackage[paperheight=19in,paperwidth=8.5in,top=1in,bottom=1in,right=1in,left=1in]{geometry}

\usepackage{fancyhdr}

\fancypagestyle{style}{
    \fancyhf{}
    \fancyhead[L]{\leftmark} %\slshape
    \fancyhead[R]{\rightmark} %\slshape
    }
\pagestyle{style}

%%%%%%%%%%%%%%%%%%%%%%%%%%%%%%%%%%%%%%%
% Tables & Figures
%%%%%%%%%%%%%%%%%%%%%%%%%%%%%%%%%%%%%%%

\usepackage{multirow,rotating,booktabs,dcolumn,longtable}   % for table cells to span rows
\usepackage[table]{xcolor} % Colored table cells
\usepackage{caption}
\usepackage{subcaption} % Captions on subfigures
\usepackage{threeparttable}
\usepackage{wrapfig}
\usepackage{multirow} % Span multiple rows in tables

%%%%%%%%%%%%%%%%%%%%%%%%%%%%%%%%%%%%%%%
% Mathematics
%%%%%%%%%%%%%%%%%%%%%%%%%%%%%%%%%%%%%%%

\usepackage{graphicx}
\usepackage{amssymb,amsmath,bm}
\usepackage{amsfonts} % Math fonts
\usepackage{amsthm} % Theorem handling
\usepackage{mathrsfs} % Math fonts
\usepackage{bbm} % For doublestruck 1
\usepackage{cancel} % Cancel terms to value in math mode
\usepackage{array} % Arrays
\usepackage{gensymb} % General symbols
\usepackage{mathdots} % Inverse diagonal dots \iddots
\usepackage{arydshln} % Dotted lines in matrices
% Example dotted line matrix:
%$\left(\,
%\begin{array}{ : c : c : }\hdashline
%\sin x & \cos x\\\hdashline
%     B & A       \\\hdashline
%\end{array}\,\right)
%$
% Should output
% (-------------------)
% ( | sin x | cos x | )
% (-------------------)
% ( |   B   |   A   | )
% (-------------------)
\usepackage{accents}
\usepackage{tensor} % For properly formatted indices on tensor objects

%%%%%%%%%%%%%%%%%%%%%%%%%%%%%%%%%%%%%%%
% Code Formatting
%%%%%%%%%%%%%%%%%%%%%%%%%%%%%%%%%%%%%%%

%\usepackage{algorithmicx}
\usepackage[linesnumbered,ruled,vlined]{algorithm2e}

\usepackage{verbatim}
\usepackage{fancyvrb}
\usepackage{listings}
\usepackage{minted}
\usepackage{upquote}
\definecolor{codegrey}{gray}{0.95}

% Formats Python code with Minted
\newminted{python}{%
frame=lines,
framesep=2mm,
baselinestretch=1.2,
bgcolor=codegrey,
fontsize=\scriptsize,
linenos}

\newenvironment{mintypython}
{\VerbatimEnvironment\begin{center}\begin{minipage}{\textwidth}\begin{pythoncode}}
{\end{pythoncode}\end{minipage}\end{center}}

%\renewcommand*\listingscaption{Code Snippet}
%\renewcommand*\listoflistingscaption{List of Code Snippets}

%%%%%%%%%%%%%%%%%%%%%%%%%%%%%%%%%%%%%%%
% List Options
%%%%%%%%%%%%%%%%%%%%%%%%%%%%%%%%%%%%%%%

\usepackage[shortlabels]{enumitem} % Enumerate with different delimiters

% Tighten up enumerate
\setlist[enumerate]{nolistsep}
%\setlist[enumerate,1]{nolistsep}
\setlist[enumerate,1]{label=(\roman*),align=left}
%\setlist[enumerate,2]{nolistsep}
\setlist[enumerate,2]{label=(\alph*),align=left}

% Add outline package, redefine delimiters
\usepackage{outlines}
\setenumerate[1]{label=\Roman*.}
\setenumerate[2]{label=\Alph*.}
\setenumerate[3]{label=\roman*.}
\setenumerate[4]{label=\alph*.}

% The below will create a 4-level I., A., i., a. outline:

%\begin{outline}[enumerate]
%   \1 Level 1
%      \2 Level 2
%         \3 Level 3
%            \4 Level 4
%\end{outline}

%%%%%%%%%%%%%%%%%%%%%%%%%%%%%%%%%%%%%%%
% Bibliography, References, Footnotes
%%%%%%%%%%%%%%%%%%%%%%%%%%%%%%%%%%%%%%%

\bibliographystyle{unsrt}
\usepackage[superscript]{cite}
%% The next two packages must be loaded in this order for footnote links to work!!
\usepackage[bottom]{footmisc} % Footnotes at the bottom
\usepackage{hyperref}  % for hyperlinks in doc. Remember...load after footmisc!

\renewcommand*{\thefootnote}{\fnsymbol{footnote}}
\usepackage{perpage}
\MakePerPage{footnote}

%%%%%%%%%%%%%%%%%%%%%%%%%%%%%%%%%%%%%%%
% Misc
%%%%%%%%%%%%%%%%%%%%%%%%%%%%%%%%%%%%%%%

\usepackage[T1]{fontenc}
\usepackage{color}

